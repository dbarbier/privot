% Copyright (c)  2010-2013  EADS.
% Permission is granted to copy, distribute and/or modify this document
% under the terms of the GNU Free Documentation License, Version 1.2
% or any later version published by the Free Software Foundation;
% with no Invariant Sections, no Front-Cover Texts, and no Back-Cover
% Texts.  A copy of the license is included in the section entitled "GNU
% Free Documentation License".

%%%%%%%%%%%%%%%%%%%%%%%%%%%%%%%%%%%%%%%%%%%%%%%%%%%%%%%%%%%%%%%%%%%%%%%%%%%%%%%%%%%%%%%%%%
\section{User Manual}

This section gives an exhaustive presentation of the objects and functions provided by the \texttt{Multivariate\-Random\-Mixture} module.

\subsection{PythonMultivariateRandomMixture}


\begin{description}
\item[Usage:]  \strut
  \begin{description}
  \item \texttt{PythonMultivariateRandomMixture(collection, matrix)}
  \item \texttt{PythonMultivariateRandomMixture(collection, matrix, vector)}
  \end{description}
\item[Arguments:] \strut
  \begin{description}
  \item \textit{collection}: a \texttt{DistributionCollection}, a collection of OpenTURNS distributions.
  \item \textit{matrix}: a \texttt{Matrix}, the matrix of the affine transformation.
  \item \textit{vector}: a \texttt{NumericalPoint}, the constant term of the affine transformation.
  \end{description}
\item[Some methods:]  \strut

  \begin{description}

  \item \texttt{computeCharacteristicFunction}
    \begin{description}
    \item[Usage:]  \texttt{computeCharacteristicFunction(u)}
    \item[Arguments:]  \strut
    \begin{description}
      \item $u$: a \texttt{NumericalPoint} of dimension $n \ge 1$
    \end{description}
    \item[Value:] a \texttt{NumericalComplex}, compute the characteristic function on $u$
    \end{description}

  \item \texttt{computeLogCharacteristicFunction}
    \begin{description}
    \item[Usage:]  \texttt{computeLogCharacteristicFunction(u)}
    \item[Arguments:]  \strut
    \begin{description}
      \item $u$: a \texttt{NumericalPoint} of dimension $n \ge 1$
    \end{description}
    \item[Value:] a \texttt{NumericalComplex}, compute the log-characteristic function on $u$
    \end{description}

  \item \texttt{computePDF}
    \begin{description}
    \item[Usage:]  \texttt{computePDF(u)}
    \item[Arguments:]  \strut
    \begin{description}
      \item $u$: a \texttt{NumericalPoint} of dimension $n \ge 1$
    \end{description}
    \item[Value:] a \texttt{NumericalScalar}, compute the probability density function on $u$
    \end{description}

  \item \texttt{getAlpha}
    \begin{description}
    \item[Usage:]  \texttt{getAlpha()}
    \item[Arguments:]  \strut
    \begin{description}
      \item \texttt{None}
    \end{description}
    \item[Value:] a \texttt{NumericalScalar}, the $\alpha$ parameter used for the computation of the range (position indicator)
    \end{description}

  \item \texttt{getBeta}
    \begin{description}
    \item[Usage:]  \texttt{getBeta()}
    \item[Arguments:]  \strut
    \begin{description}
      \item \texttt{None}
    \end{description}
    \item[Value:] a \texttt{NumericalScalar}, the $\beta$ parameter used for the computation of the range (dispersion indicator)
    \end{description}

  \item \texttt{getBlockMax}
    \begin{description}
    \item[Usage:]  \texttt{getBlockMax()}
    \item[Arguments:]  \strut
    \begin{description}
      \item \texttt{None}
    \end{description}
    \item[Value:] an \texttt{UnsignedLong}, the maximal parameter size (i.e $2^{\text{blockMax}}$) used for the computation of the differences of characteristic functions
    \end{description}

  \item \texttt{getBlockMin}
    \begin{description}
    \item[Usage:]  \texttt{getBlockMin()}
    \item[Arguments:]  \strut
    \begin{description}
      \item \texttt{None}
    \end{description}
    \item[Value:] an \texttt{UnsignedLong}, the minimal parameter size (i.e $2^{\text{blockMin}}$) used for the computation of the differences of characteristic functions
    \end{description}

  \item \texttt{getConstant}
    \begin{description}
    \item[Usage:]  \texttt{getConstant()}
    \item[Arguments:]  \strut
    \begin{description}
      \item \texttt{None}
    \end{description}
    \item[Value:] a \texttt{NumericalPoint}, the deterministic vector of the transformation
    \end{description}

  \item \texttt{getCorrelation}
    \begin{description}
    \item[Usage:]  \texttt{getCorrelation()}
    \item[Arguments:]  \strut
    \begin{description}
      \item \texttt{None}
    \end{description}
    \item[Value:] a \texttt{CorrelationMatrix}, the correlation of the distribution
    \end{description}

  \item \texttt{getCovariance}
    \begin{description}
    \item[Usage:]  \texttt{getCovariance()}
    \item[Arguments:]  \strut
    \begin{description}
      \item \texttt{None}
    \end{description}
    \item[Value:] a \texttt{CovarianceMatrix}, the covariance of the distribution
    \end{description}

  \item \texttt{getDimension}
    \begin{description}
    \item[Usage:]  \texttt{getDimension()}
    \item[Arguments:]  \strut
    \begin{description}
      \item \texttt{None}
    \end{description}
    \item[Value:] an \texttt{UnsignedLong}, the dimension of the matrix
    \end{description}

  \item \texttt{getDistributionCollection}
    \begin{description}
    \item[Usage:]  \texttt{getDistributionCollection()}
    \item[Arguments:]  \strut
    \begin{description}
      \item \texttt{None}
    \end{description}
    \item[Value:] a \texttt{DistributionCollection}, the collection of distributions used for the affine transformation
    \end{description}

  \item \texttt{getLastPDFError}
    \begin{description}
    \item[Usage:]  \texttt{getLastPDFError()}
    \item[Arguments:]  \strut
    \begin{description}
      \item \texttt{None}
    \end{description}
    \item[Value:] a \texttt{NumericalScalar}, the last error obtained when computing the probability density function
    \end{description}

  \item \texttt{getMatrix}
    \begin{description}
    \item[Usage:]  \texttt{getMatrix()}
    \item[Arguments:]  \strut
    \begin{description}
      \item \texttt{None}
    \end{description}
    \item[Value:] a \texttt{Matrix}, the \emph{weight} matrix used for the affine transformation
    \end{description}

  \item \texttt{getMaxSize}
    \begin{description}
    \item[Usage:]  \texttt{getMaxSize()}
    \item[Arguments:]  \strut
    \begin{description}
      \item \texttt{None}
    \end{description}
    \item[Value:] an \texttt{UnsignedLong}, the maximum size of the cache for the CharacteristicFunction values
    \end{description}

  \item \texttt{getMean}
    \begin{description}
    \item[Usage:]  \texttt{getMean()}
    \item[Arguments:]  \strut
    \begin{description}
      \item \texttt{None}
    \end{description}
    \item[Value:] a \texttt{NumericalPoint}, the mean of the distribution
    \end{description}

  \item \texttt{getPDFPrecision}
    \begin{description}
    \item[Usage:]  \texttt{getPDFPrecision()}
    \item[Arguments:]  \strut
    \begin{description}
      \item \texttt{None}
    \end{description}
    \item[Value:] a \texttt{NumericalScalar}, the precision used for the computation of the probability density function
    \end{description}

  \item \texttt{getRange}
    \begin{description}
    \item[Usage:]  \texttt{getRange()}
    \item[Arguments:]  \strut
    \begin{description}
      \item \texttt{None}
    \end{description}
    \item[Value:] an \texttt{Interval} of dimension $d$, the range of the distribution
    \end{description}

  \item \texttt{getRealization}
    \begin{description}
    \item[Usage:]  \texttt{getRealization()}
    \item[Arguments:]  \strut
    \begin{description}
      \item \texttt{None}
    \end{description}
    \item[Value:] a \texttt{NumericalPoint}, a realization of the distribution
    \end{description}

  \item \texttt{getReferenceBandwidth}
    \begin{description}
    \item[Usage:]  \texttt{getReferenceBandwidth()}
    \item[Arguments:]  \strut
    \begin{description}
      \item \texttt{None}
    \end{description}
    \item[Value:] a \texttt{NumericalPoint}, the bandwidth used for the computation of the probability density function
    \end{description}

  \item \texttt{getSample}
    \begin{description}
    \item[Usage:]  \texttt{getSample()}
    \item[Arguments:]  \strut
    \begin{description}
      \item \texttt{n}: a positive integer
    \end{description}
    \item[Value:] a \texttt{NumericalSample}, a sample of $n$ realizations of the distribution
    \end{description}

  \item \texttt{getStandardDeviation}
    \begin{description}
    \item[Usage:]  \texttt{getStandardDeviation()}
    \item[Arguments:]  \strut
    \begin{description}
      \item \texttt{None}
    \end{description}
    \item[Value:] a \texttt{NumericalPoint}, the standard deviation of the distribution
    \end{description}

  \item \texttt{setAlpha}
    \begin{description}
    \item[Usage:]  \texttt{setAlpha(alpha)}
    \item[Arguments:]  \strut
    \begin{description}
      \item $alpha$, a \texttt{NumericalScalar}
    \end{description}
    \item[Value:]  \texttt{None}, sets the $\alpha$ parameter used for the computation of the range (position indicator)
    \end{description}

  \item \texttt{setBeta}
    \begin{description}
    \item[Usage:] \texttt{setBeta(beta)}
    \item[Arguments:] \strut
    \begin{description}
      \item $beta$: a \texttt{NumericalScalar}
    \end{description}
    \item[Value:] \texttt{None}, sets the $\beta$ parameter used for the computation of the range (dispersion indicator)
    \end{description}

  \item \texttt{setBlockMax}
    \begin{description}
    \item[Usage:]  \texttt{setBlockMax(blockMax)}
    \item[Arguments:]  \strut
    \begin{description}
      \item $blockMax$, an \texttt{UnsignedLong}
    \end{description}
    \item[Value:] \texttt{None}, sets the maximal parameter size (i.e $2^{\text{blockMax}}$) used for the computation of the differences of characteristic functions
    \end{description}

  \item \texttt{setBlockMin}
    \begin{description}
    \item[Usage:]  \texttt{setBlockMin(blockMin)}
    \item[Arguments:]  \strut
    \begin{description}
      \item $blockMin$, an \texttt{UnsignedLong}
    \end{description}
    \item[Value:] \texttt{None}, sets the minimal parameter size (i.e $2^{\text{blockMin}}$) used for the computation of the differences of characteristic functions
    \end{description}

  \item \texttt{setMaxSize}
    \begin{description}
    \item[Usage:]  \texttt{setMaxSize(maxSize)}
    \item[Arguments:]  \strut
    \begin{description}
      \item $maxSize$, an \texttt{UnsignedLong}
    \end{description}
    \item[Value:] \texttt{None}, sets the maximum size of the cache for the CharacteristicFunction values
    \end{description}

  \item \texttt{setReferenceBandwidth}
    \begin{description}
    \item[Usage:]  \texttt{setReferenceBandwidth(bandwidth)}
    \item[Arguments:] \strut
    \begin{description}
      \item $bandwidth$, a \texttt{NumericalPoint}
    \end{description}
    \item[Value:] \texttt{None}, sets the bandwidth used for the computation of the probability density function
    \end{description}

  \item \texttt{setPDFPrecision}
    \begin{description}
    \item[Usage:]  \texttt{setPDFPrecision()}
    \item[Arguments:]  \strut
    \begin{description}
      \item $pdfPrecision$, a \texttt{NumericalScalar}
    \end{description}
    \item[Value:] \texttt{None}, sets the precision used for the computation of the probability density function
    \end{description}

  \end{description}

\end{description}

\subsection{MultivariateRandomMixture}
This class inherits from the \texttt{Distribution} class.

\begin{description}
\item[Usage:]  \strut
  \begin{description}
  \item \texttt{MultivariateRandomMixture(collection, matrix)}
  \item \texttt{MultivariateRandomMixture(collection, matrix, vector)}
  \end{description}
\item[Arguments:] \strut
  \begin{description}
  \item \textit{collection}: a \texttt{DistributionCollection}, a collection of OpenTURNS distributions.
  \item \textit{matrix}: a \texttt{Matrix}, the matrix of the affine transformation.
  \item \textit{vector}: a \texttt{NumericalPoint}, the constant term of the affine transformation.
  \end{description}
\end{description}
