% Copyright (c)  2010-2013  EADS.
% Permission is granted to copy, distribute and/or modify this document
% under the terms of the GNU Free Documentation License, Version 1.2
% or any later version published by the Free Software Foundation;
% with no Invariant Sections, no Front-Cover Texts, and no Back-Cover
% Texts.  A copy of the license is included in the section entitled "GNU
% Free Documentation License".




%%%%%%%%%%%%%%%%%%%%%%%%%%%%%%%%%%%%%%%%%%%%%%%%%%%%%%%%%%%%%%%%%%%%%%%%%%%%%%%%%%%%%%%%%%
\section{Reference Guide}

\subsection{Multivariate random mixtures}

The objective is to manipulate a multivariate random vector $\vect{Y}$ with a specific structure:
it is the image of a random vector $\vect{X} = (X_1, \hdots, X_n) $ of independent and continuous univariate distributions by an affine transform $T$:
\begin{align}\label{defY}
  \vect{Y}=\vect{y}_0+\mat{M}\vect{X}
\end{align}
$\vect{y}_0\in\mathbb{R}^d$ is a deterministic vector with $d\in\{1,2,3\}$ and $\vect{M}\in\mathcal{M}_{d,n}(\mathbb{R})$
a deterministic matrix. This particular form allows the calculation of the law of $\vect{Y} $ using the Poisson summation formula, which links the characteristic
function $\phi $ of $\vect{Y}$ to its density $p$. This formula is given by:

\begin{align}\label{poissonND}
  \sum_{j_1\in\mathbb{Z}}\hdots\sum_{j_d\in\mathbb{Z}} p\left(y_1+\frac{2\pi j_1}{h_1},\right.&\left.\hdots,y_d+\frac{2\pi j_d}{h_d}\right)=\notag\\
    &\frac{h_1\times\cdots\times h_d}{2^d\pi^d}\sum_{k_1\in\mathbb{Z}}\hdots\sum_{k_d\in\mathbb{Z}}\phi\left(k_1h_1,\hdots,k_dh_d\right)e^{-i(k_1h_1y_1+\cdots+k_dh_dy_d)}
\end{align}

By fixing $h_1,\hdots,h_d$ enough small, the nested sums of the left term are reduced to the central term $j_1=\hdots=j_d = 0$.
The sums on the right are truncated symmetrically to $ 2N +1 $ terms, which gives:
\begin{align}
  p\left(y_1,\hdots,y_d\right)\simeq\frac{H}{2^d\pi^d}\sum_{|k_1|\leq N}\hdots\sum_{|k_d|\leq N}\phi\left(k_1h_1,\hdots,k_dh_d\right)e^{-i(k_1h_1y_1+\cdots+k_dh_dy_d)}
\end{align}
where $H = h_1\times\hdots\times h_d$.

The characteristic function of such a random vector $Y$ is obtained analytically using $\vect{y}_0$, $\vect{M}$ and the characteristic functions of $X_1$,\dots,$X_n$:
\begin{align}
  \forall \vect{u}\in\mathbb{R}^d,\quad\phi(u_1,\hdots,u_d)=\prod_{j=1}^de^{iu_j{y_0}_j}\prod_{k=1}^n\phi_{X_k}((M^tu)_k)
\end{align}

It is possible to greatly improve the performance of the algorithm by noticing that equation~(\ref{poissonND}) is linear between $p$
and $\phi $. We denote $q$ and $\psi$ respectively the density and the characteristic function of the multivariate normal distribution with the
same mean $\vect{\mu}$ and same covariance matrix $\vect{C}$ as the the random mixture. By applying this multivariate normal distribution to the equation~(\ref{poissonND}), we obtain by subtraction:
\begin{align}\label{algoPoisson}
p\left(y_1,\hdots,y_d\right)\simeq \sum_{j_1\in\mathbb{Z}}\hdots\sum_{j_d\in\mathbb{Z}}
  &q\left(y_1+\frac{2\pi j_1}{h_1},\hdots,y_d+\frac{2\pi j_d}{h_d}\right)+\notag\\
  &\frac{H}{2^d\pi^d}\sum_{|k_1|\leq N}\hdots\sum_{|k_d|\leq N}(\phi-\psi)\left(k_1h_1,\hdots,k_dh_d\right)e^{-i(k_1h_1y_1+\cdots+k_dh_dy_d)}
\end{align}

The vector $\vect{\mu}$ and the covariance matrix $\mat{C}$ of the random vector $\vect{Y}$ are obtained using $\vect{X}$ and the following formula:
\begin{align}
  \vect{\mu}=& \vect{y_0} + \mat{M}\mathbb{E}(\vect{X})\\
  \mat{C}=&\mat{M}\,\mat{\mathrm{Cov}}(\vect{X})\mat{M}^t\\
\end{align}

The matrix $\mat{\mathrm{Cov}}(\vect{X})$ is diagonal thanks to the the independence of $X_i$.
The element $C_{i,j}$ of $\mat{C}$ is given by:
\begin{align}
\forall i,j\in\{1,\hdots,d\},\quad C_{i,j}=\sum_{k=1}^nM_{i,k}M_{j,k}\mathrm{Var}(X_k)
\end{align}

In the case where $n \gg $ 1, using the limit central theorem, the law of $\vect{Y} $ tends to the normal distribution density $q$,
which will drastically reduce $N$.  The sum on $q$ will become the most CPU-intensive part, because in the general case we will
have to keep more terms than the central one in this sum, since the parameters $ h_1, \dots  h_d$ were calibrated
with respect to $p$ and not $q$.

The parameters $h_1, \dots  h_d$ are calibrated using the following formula:
\begin{align}
  h_\ell = \frac{2\pi}{(b+4a)\sigma_\ell}
\end{align}
where $\sigma_\ell=\sqrt{C_{\ell,\ell}}$ and $a$, $b$ are respectively the number of standard deviations covered by the marginal distribution
($a=4$ by default) and $b$ the number of marginal deviations beyond which the density is negligible ($b=8$ by default).

The $N$ parameter is dynamically calibrated: we start with $N=8$ then we double $N$ value until the total contribution of the additional terms is negligible.

\subsection{References}\label{ref}
\begin{itemize}
  \item[1] "Abate, J. and Whitt, W. (1992). The Fourier-series method for inverting transforms of probability distributions. Queueing Systems 10, 5--88., 1992",
        formula 5.5.
\end{itemize}

\section{Computation on a regular grid}
\addtocontents{toc}{\protect\setcounter{tocdepth}{2}}

The interest is to compute the density function on a regular grid. Purposes are drawing some iso-values.
The regular grid is of form:
\begin{align}
  \forall r\in\{1,\hdots,d\},\forall m\in\{0,\hdots,M-1\},\:y_{r,m}=\mu_r+b\left(\frac{2m+1}{M} - 1\right)\sigma_r
\end{align}

By denoting $p_{m_1,\hdots,m_d}=p_{\vect{Y}}(y_{1,m_1},\hdots,y_{d,m_d})$:
\begin{align}
  p_{m_1,\hdots,m_d}= Q_{m_1,\hdots,m_d}+S_{m_1,\hdots,m_d}
\end{align}
for which the term $S_{m_1,\hdots,m_d}$ is the most CPU consuming. This term rewrites:
\begin{align}
S_{m_1,\hdots,m_d}=&\frac{H}{2^d\pi^d}\sum_{k_1=-N}^{N}\hdots\sum_{k_d=-N}^{N}\delta\left(k_1h_1,\hdots,k_dh_d\right)
E_{m_1,\hdots,m_d}(k_1,\hdots,k_d)
\end{align}
with:
\begin{align}
  \delta\left(k_1h_1,\hdots,k_dh_d\right)&=(\phi-\psi)\left(k_1h_1,\hdots,k_dh_d\right)\\
  E_{m_1,\hdots,m_d}(k_1,\hdots,k_d)&=e^{-i\sum_{j=1}^d k_jh_j\left(\mu_j+b\left(\frac{2m_j+1}{M}-1\right)\sigma_j\right)}
\end{align}

The aim is to rewrite the previous expression as a $d$- discrete Fourier transform, in order to apply Fast Fourier Transform (\emph{FFT}) for its evaluation.
We should before that divide the sums of the expression into blocks of optimal sizes for the FFT. This part of job is complex and some key issues are resumed here after.

\subsection{The unidimensional case}

The interest is to compute
\begin{align}
S_{m}=\frac{h}{2\pi}\Big\{&\Sigma_{m}^{+} + \Sigma_{m}^{-} \Big\}
\end{align}
with:
\begin{align}
\Sigma_{m}^{+}&=\sum_{k=1}^{N}\delta( kh)E_{m}(k)\\
\Sigma_{m}^{-}&=\sum_{k=1}^{N}\delta(-kh)E_{m}(-k)\\
\end{align}

\subsubsection{$\Sigma_{m}^{+}$}
By developing $E_m(k)$, we get:
\begin{align*}
E_{m}(k)&=e^{-ikh\left(\mu+b\left(\frac{2m+1}{M}-1\right)\sigma\right)}\\
&=e^{-ikhb\sigma\left(\frac{\mu}{b\sigma}+\frac{2m}{M}+\frac{1}{M}-1\right)}\\
\end{align*}
We set $M=N$, $h=\frac{\pi}{b\sigma}$ and $\tau=\frac{\mu}{b\sigma}$. We obtain:
\begin{align*}
E_{m}(k)&=e^{-2i\pi\left(\frac{k m}{N}\right)}e^{-i\pi k\left(\tau-1+\frac{1}{N}\right)}
\end{align*}
By shifting sum index,
\begin{align*}
\Sigma_{m}^{+}&=\sum_{k=0}^{N-1}\delta((k+1)h)E_{m}(k+1)\\
  &=\sum_{k=0}^{N-1}\delta((k+1)h) e^{-2i\pi\left(\frac{(k+1) m}{N}\right)} e^{-i\pi (k+1)\left(\tau-1+\frac{1}{N}\right)}\\
  &=e^{-2i\pi\left(\frac{m}{N}\right)} \sum_{k=0}^{N-1}\left(\delta((k+1)h) e^{-i\pi (k+1)\left(\tau-1+\frac{1}{N}\right)}\right) e^{-2i\pi\left(\frac{k m}{N}\right)} 
\end{align*}
one can identify a discrete Fourier transform.

\begin{enumerate}
\item For $k\in\{0,\hdots,N-1\}$, we set:
\begin{align}
y_{k}=\delta((k+1)h)e^{-i\pi (k+1)\left(\tau-1+\frac{1}{N}\right)}
\end{align}
\item For $m \in\{0,\hdots,N-1\}$, we set:
\begin{align}
  z_{m}=fft(y_{k})
\end{align}
with $fft$ a FFT algorithm
\item Eventually $\Sigma_{m}^{+}$ is:
\begin{align*}
  \Sigma_{m}^{+}=z_{m}e^{-2i\pi\left(\frac{m}{N}\right)}
\end{align*}
\end{enumerate}

\subsubsection{$\Sigma_{m}^{-}$}

We apply a variable change $k \rightarrow k-N$ and use the symmetry of the $\delta$ function
\begin{align*}
\delta(-kh)=&\bar{\delta}(kh)
\end{align*}
in order to reuse values computed in the previous section:
\begin{align*}
\Sigma_{m}^{-}&=\sum_{k=1}^{N}\delta(-kh)E_{m}(-k)\\
  &=\sum_{k=0}^{N-1}\delta((k-N)h)E_{m}(k-N)\\
  &=\sum_{k=0}^{N-1}\delta((k-N)h) e^{-2i\pi\left(\frac{(k-N) m}{N}\right)}e^{-i\pi (k-N)\left(\tau-1+\frac{1}{N}\right)}\\
  &=\sum_{k=0}^{N-1}\left(\overline{\delta((N-k)h) e^{-i\pi (N-k)\left(\tau-1+\frac{1}{N}\right)}}\right) e^{-2i\pi\left(\frac{k m}{N}\right)} 
\end{align*}

We can again identify a discrete Fourier transform, and reuse $y_k$ computed previously.
\begin{enumerate}
\item Pour $k\in\{0,\hdots,N-1\}$,
\begin{align}
z_{k}= \bar{y}_{N-1-k}
\end{align}
\item For $m \in\{0,\hdots,N-1\}$,
\begin{align}
  \Sigma_{m}^{-}=fft(z_{k})
\end{align}
with $fft$ a FFT algorithm
\end{enumerate}

\subsection{The bidimensional case}
In dimension 2, the expression of interest is:
\begin{align}
S_{m_1,m_2}=\frac{h_1h_2}{4\pi^2}\Big\{
  \Sigma_{m_1,m_2}^{++} + \Sigma_{m_1,m_2}^{--} + \Sigma_{m_1,m_2}^{+-} + \Sigma_{m_1,m_2}^{-+} +
  \Sigma_{m_1,m_2}^{+0} + \Sigma_{m_1,m_2}^{-0} + \Sigma_{m_1,m_2}^{0+} + \Sigma_{m_1,m_2}^{0-}
\Big\}
\end{align}
with:
\begin{align}
\Sigma_{m_1,m_2}^{++}&=\sum_{k_1=1}^{N}\sum_{k_2=1}^{N}\delta\left(k_1h_1,k_2h_2\right)E_{m_1,m_2}(k_1,k_2)\\
\Sigma_{m_1,m_2}^{--}&=\sum_{k_1=1}^{N}\sum_{k_2=1}^{N}\delta\left(-k_1h_1,-k_2h_2\right)E_{m_1,m_2}(-k_1,-k_2)\\
\Sigma_{m_1,m_2}^{+-}&=\sum_{k_1=1}^{N}\sum_{k_2=1}^{N}\delta\left(k_1h_1,-k_2h_2\right)E_{m_1,m_2}(k_1,-k_2)\\
\Sigma_{m_1,m_2}^{-+}&=\sum_{k_1=1}^{N}\sum_{k_2=1}^{N}\delta\left(-k_1h_1,k_2h_2\right)E_{m_1,m_2}(-k_1,k_2)\\
\Sigma_{m_1,m_2}^{+0}&=\sum_{k_1=1}^{N}\delta(k_1h_1,0)E_{m_1}(k_1)\\
\Sigma_{m_1,m_2}^{-0}&=\sum_{k_1=1}^{N}\delta(-k_1h_1,0)E_{m_1}(-k_1)\\
\Sigma_{m_1,m_2}^{0+}&=\sum_{k_2=1}^{N}\delta(0,k_2h_2)E_{m_2}(k_2)\\
\Sigma_{m_1,m_2}^{0-}&=\sum_{k_2=1}^{N}\delta(0,-k_2h_2)E_{m_2}(-k_2)
\end{align}
We use the same techniques as in dimension~1 (symmetries, variable changes) in order to accurately evaluate the $\delta$ applied to the couples $(k_1, k_2)$ with $k_1\geq 0$:
\begin{align}
\delta\left(-k_1h_1,-k_2h_2\right)=&\bar{\delta}\left(k_1h_1,k_2h_2\right)\\
\delta\left(-k_1h_1, k_2h_2\right)=&\bar{\delta}\left(k_1h_1,-k_2h_2\right)
\end{align}

\subsubsection{$\Sigma_{m_1,m_2}^{++}$}

\begin{align*}
E_{m_1,m_2}(k_1,k_2)
 &= e^{-ik_1h_1b\sigma_1\left(\frac{\mu_1}{b\sigma_1}+\frac{2m_1}{M}+\frac{1}{M}-1\right)}
    e^{-ik_2h_2b\sigma_2\left(\frac{\mu_2}{b\sigma_2}+\frac{2m_2}{M}+\frac{1}{M}-1\right)}
\end{align*}
We fix $M=N$, $h_1=\frac{\pi}{b\sigma_1}$, $h_2=\frac{\pi}{b\sigma_2}$, $\tau_1=\frac{\mu_1}{b\sigma_1}$, $\tau_2=\frac{\mu_2}{b\sigma_2}$ and obtain
\begin{align*}
E_{m_1,m_2}(k_1,k_2)
 &= e^{-ik_1\pi\left(\tau_1+\frac{2m_1}{N}+\frac{1}{N}-1\right)}
    e^{-ik_2\pi\left(\tau_2+\frac{2m_2}{N}+\frac{1}{N}-1\right)}\\
 &= e^{-2i\pi\left(\frac{k_1m_1+k_2m_2}{N}\right)}
    e^{-ik_1\pi\left(\tau_1+\frac{1}{N}-1\right)}
    e^{-ik_2\pi\left(\tau_2+\frac{1}{N}-1\right)}
\end{align*}

And thus
\begin{align*}
\Sigma_{m_1,m_2}^{++}=e^{-2i\pi\left(\frac{m_1+m_2}{N}\right)}\sum_{k_1=0}^{N-1}\sum_{k_2=0}^{N-1}
  &\delta\left((k_1+1)h_1,(k_2+1)h_2\right) \times\\
  & e^{-2i\pi\left(\frac{k_1m_1+k_2m_2}{N}\right)}
    e^{-i(k_1+1)\pi\left(\tau_1+\frac{1}{N}-1\right)}
    e^{-i(k_2+1)\pi\left(\tau_2+\frac{1}{N}-1\right)}
\end{align*}

As in the unidimensional case, the evaluation of $\Sigma^{++}$ might be done using a FFT algorithm:
\begin{enumerate}
\item For $k_1,k_2\in\{0,\hdots,N-1\}$, we fix:
\begin{align*}
y^{++}_{k_1,k_2}=\delta((k_1+1)h_1,(k_2+1)h_2)e^{-i\pi (k_1+1)\left(\tau_1-1+\frac{1}{N}\right)}e^{-i\pi (k_2+1)\left(\tau_2-1+\frac{1}{N}\right)}
\end{align*}
\item For $m_1,m_2\in\{0,\hdots,N-1\}$, we set:
\begin{align*}
  z_{m_1,m_2}=fft2d(y^{++}_{k_1,k_2})
\end{align*}
with $fft2d$ a bidimensional discrete Fourier transform.
\item The contribution $\Sigma_{m_1,m_2}^{++}$ equals to:
\begin{align*}
  \Sigma_{m_1,m_2}^{++}=z_{m_1,m_2}e^{-2i\pi\left(\frac{m_1+m_2}{N}\right)}
\end{align*}
\end{enumerate}

\subsubsection{$\Sigma_{m_1,m_2}^{--}$}

\begin{enumerate}
\item For $k_1,k_2\in\{0,\hdots,N-1\}$, we fix:
\begin{align*}
y^{--}_{k_1,k_2}&=\bar{y}^{++}_{N-1-k_1,N-1-k_2}
\end{align*}
\item The contribution $\Sigma_{m_1,m_2}^{--}$ equals to:
\begin{align*}
  \Sigma_{m_1,m_2}^{--}=fft2d(y^{--}_{k_1,k_2})
\end{align*}
\end{enumerate}

\subsubsection{$\Sigma_{m_1,m_2}^{+-}$}

\begin{enumerate}
\item For $k_1,k_2\in\{0,\hdots,N-1\}$, we fix:
\begin{align*}
y^{+-}_{k_1,k_2}= \delta((k_1+1)h_1,(k_2-N)h_2)e^{-i\pi (k_1+1)\left(\tau_1-1+\frac{1}{N}\right)}e^{-i\pi (k_2-N)\left(\tau_2-1+\frac{1}{N}\right)}
\end{align*}
\item For $m_1,m_2\in\{0,\hdots,N-1\}$, we set:
\begin{align*}
  z_{m_1,m_2}=fft2d(y^{+-}_{k_1,k_2})
\end{align*}
with $fft2d$ a bidimensional Fourier transform.
\item The contribution $\Sigma_{m_1,m_2}^{+-}$ equals to:
\begin{align*}
  \Sigma_{m_1,m_2}^{+-}=z_{m_1,m_2}e^{-2i\pi\left(\frac{m_1}{N}\right)}
\end{align*}
\end{enumerate}


\subsubsection{$\Sigma_{m_1,m_2}^{-+}$}

\begin{enumerate}
\item For $k_1,k_2\in\{0,\hdots,N-1\}$, we fix:
\begin{align*}
y^{-+}_{k_1,k_2}&=\bar{y}^{+-}_{N-1-k_1,N-1-k_2}
\end{align*}
\item For $m_1,m_2\in\{0,\hdots,N-1\}$, we set:
\begin{align*}
  z_{m_1,m_2}=fft2d(y^{-+}_{k_1,k_2})
\end{align*}
with $fft2d$ a bidimensional Fourier transform.
\item The contribution $\Sigma_{m_1,m_2}^{-+}$ equals to:
\begin{align*}
  \Sigma_{m_1,m_2}^{-+}=z_{m_1,m_2} e^{-2i\pi\left(\frac{m_2}{N}\right)}
\end{align*}
\end{enumerate}


\subsubsection{$\Sigma_{m_1,m_2}^{+0}$}
\begin{enumerate}
\item For $k_1\in\{0,\hdots,N-1\}$, we fix:
\begin{align*}
y^{+0}_{k_1}= \delta((k_1+1)h_1,0)e^{-i\pi (k_1+1)\left(\tau_1-1+\frac{1}{N}\right)}
\end{align*}
\item For $m_1\in\{0,\hdots,N-1\}$, we set:
\begin{align*}
  z_{m_1}=fft1d(y^{+0}_{k_1})
\end{align*}
with $fft1d$ a unidimensional Fourier transform.
\item The contribution $\Sigma_{m_1,m_2}^{+0}$ equals to:
\begin{align*}
  \Sigma_{m_1,m_2}^{+0}=z_{m_1} e^{-2i\pi\left(\frac{m_1}{N}\right)}
\end{align*}
\end{enumerate}

\subsubsection{$\Sigma_{m_1,m_2}^{-0}$}
\begin{enumerate}
\item For $k_1\in\{0,\hdots,N-1\}$, we fix:
\begin{align*}
y^{-0}_{k_1}= \bar{y}^{+0}_{N-1-k_1}
\end{align*}
\item The contribution $\Sigma_{m_1,m_2}^{-0}$ equals to:
\begin{align*}
  \Sigma_{m_1,m_2}^{-0}=fft1d(y^{-0}_{k_1})
\end{align*}
with $fft1d$ a unidimensional Fourier transform.
\end{enumerate}

\subsubsection{$\Sigma_{m_1,m_2}^{0+}$}
\begin{enumerate}
\item For $k_2\in\{0,\hdots,N-1\}$, we fix:
\begin{align*}
y^{0+}_{k_2}= \delta(0,(k_2+1)h_2)e^{-i\pi (k_2+1)\left(\tau_2-1+\frac{1}{N}\right)}
\end{align*}
\item For $m_2\in\{0,\hdots,N-1\}$, we set:
\begin{align*}
  z_{m_2}=fft1d(y^{0+}_{k_2})
\end{align*}
with $fft1d$ a unidimensional Fourier transform.
\item The contribution $\Sigma_{m_1,m_2}^{0+}$ equals to:
\begin{align*}
  \Sigma_{m_1,m_2}^{0+}=z_{m_2}e^{-2i\pi\left(\frac{m_2}{N}\right)}
\end{align*}
\end{enumerate}

\subsubsection{$\Sigma_{m_1,m_2}^{0-}$}
\begin{enumerate}
\item For $k_2\in\{0,\hdots,N-1\}$, we fix:
\begin{align*}
y^{0-}_{k_2}= \bar{y}^{0+}_{N-1-k_2}
\end{align*}
\item The contribution $\Sigma_{m_1,m_2}^{0-}$ equals to:
\begin{align*}
  \Sigma_{m_1,m_2}^{0-}=fft1d(y^{0-}_{k_2})
\end{align*}
with $fft1d$ a unidimensional Fourier transform.
\end{enumerate}

\subsection{The tridimensional case}
In dimension 3, the expression of interest is~:
\begin{align*}
S_{m_1,m_2,m_3}=\frac{h_1h_2h_3}{8\pi^3}\Big\{
 &   \Sigma_{m1,m2,m3}^{+++} + \Sigma_{m1,m2,m3}^{---} +
     \Sigma_{m1,m2,m3}^{++-} + \Sigma_{m1,m2,m3}^{--+} +
     \Sigma_{m1,m2,m3}^{+-+} + \Sigma_{m1,m2,m3}^{-+-} + \\
 &   \Sigma_{m1,m2,m3}^{+--} + \Sigma_{m1,m2,m3}^{-++} +
     \Sigma_{m1,m2,m3}^{++0} + \Sigma_{m1,m2,m3}^{--0} +
     \Sigma_{m1,m2,m3}^{0++} + \Sigma_{m1,m2,m3}^{0--} + \\
 &   \Sigma_{m1,m2,m3}^{+0+} + \Sigma_{m1,m2,m3}^{-0-} +
     \Sigma_{m1,m2,m3}^{+-0} + \Sigma_{m1,m2,m3}^{-+0} +
     \Sigma_{m1,m2,m3}^{+0-} + \Sigma_{m1,m2,m3}^{-0+} + \\
 &   \Sigma_{m1,m2,m3}^{0+-} + \Sigma_{m1,m2,m3}^{0-+} +
     \Sigma_{m1,m2,m3}^{+00} + \Sigma_{m1,m2,m3}^{-00} +
     \Sigma_{m1,m2,m3}^{0+0} + \Sigma_{m1,m2,m3}^{0-0} + \\
 &   \Sigma_{m1,m2,m3}^{00+} + \Sigma_{m1,m2,m3}^{00-}
\Big\}
\end{align*}

with
\begin{align*}
\Sigma_{m1,m2,m3}^{+++}&=\sum_{k1=1}^{N}\sum_{k2=1}^{N}\sum_{k3=1}^{N}\delta(k_1h_1,k_2h_2,k_3h_3) E_{m1,m2,m3}(k_1,k_2,k_3)\\
\Sigma_{m1,m2,m3}^{---}&=\sum_{k1=1}^{N}\sum_{k2=1}^{N}\sum_{k3=1}^{N}\delta(-k_1h_1,-k_2h_2,-k_3h_3) E_{m1,m2,m3}(-k_1,-k_2,-k_3)\\
\Sigma_{m1,m2,m3}^{++-}&=\sum_{k1=1}^{N}\sum_{k2=1}^{N}\sum_{k3=1}^{N}\delta(k_1h_1,k_2h_2,-k_3h_3) E_{m1,m2,m3}(k_1,k_2,-k_3)\\
\Sigma_{m1,m2,m3}^{--+}&=\sum_{k1=1}^{N}\sum_{k2=1}^{N}\sum_{k3=1}^{N}\delta(-k_1h_1,-k_2h_2,k_3h_3) E_{m1,m2,m3}(-k_1,-k_2,k_3)\\
\Sigma_{m1,m2,m3}^{+-+}&=\sum_{k1=1}^{N}\sum_{k2=1}^{N}\sum_{k3=1}^{N}\delta(k_1h_1,-k_2h_2,k_3h_3) E_{m1,m2,m3}(k_1,-k_2,k_3)\\
\Sigma_{m1,m2,m3}^{-+-}&=\sum_{k1=1}^{N}\sum_{k2=1}^{N}\sum_{k3=1}^{N}\delta(-k_1h_1,k_2h_2,-k_3h_3) E_{m1,m2,m3}(-k_1,k_2,-k_3)\\
\Sigma_{m1,m2,m3}^{+--}&=\sum_{k1=1}^{N}\sum_{k2=1}^{N}\sum_{k3=1}^{N}\delta(k_1h_1,-k_2h_2,-k_3h_3) E_{m1,m2,m3}(k_1,-k_2,-k_3)\\
\Sigma_{m1,m2,m3}^{-++}&=\sum_{k1=1}^{N}\sum_{k2=1}^{N}\sum_{k3=1}^{N}\delta(k_1h_1,-k_2h_2,-k_3h_3) E_{m1,m2,m3}(k_1,-k_2,-k_3)\\
\Sigma_{m1,m2,m3}^{++0}&=\sum_{k1=1}^{N}\sum_{k2=1}^{N}\delta(k_1h_1,k_2h_2,0) E_{m1,m2,m3}(k_1,k_2,0)\\
\Sigma_{m1,m2,m3}^{--0}&=\sum_{k1=1}^{N}\sum_{k2=1}^{N}\delta(-k_1h_1,-k_2h_2,0) E_{m1,m2,m3}(-k_1,-k_2,0)\\
\Sigma_{m1,m2,m3}^{0++}&=\sum_{k2=1}^{N}\sum_{k3=1}^{N}\delta(0,k_2h_2,k_3h_3) E_{m1,m2,m3}(0,k_2,k_3)\\
\Sigma_{m1,m2,m3}^{0--}&=\sum_{k2=1}^{N}\sum_{k3=1}^{N}\delta(0,-k_2h_2,-k_3h_3) E_{m1,m2,m3}(0,k_2,k_3)\\
\Sigma_{m1,m2,m3}^{+0+}&=\sum_{k1=1}^{N}\sum_{k3=1}^{N}\delta(k_1h_1,0,k_3h_3) E_{m1,m2,m3}(k_1,0,k_3)\\
\Sigma_{m1,m2,m3}^{-0-}&=\sum_{k1=1}^{N}\sum_{k3=1}^{N}\delta(-k_1h_1,0,-k_3h_3) E_{m1,m2,m3}(k_1,0,k_3)\\
\Sigma_{m1,m2,m3}^{+-0}&=\sum_{k1=1}^{N}\sum_{k2=1}^{N}\delta(k_1h_1,-k_2h_2, 0) E_{m1,m2,m3}(k_1,-k_2,0)\\
\Sigma_{m1,m2,m3}^{-+0}&=\sum_{k1=1}^{N}\sum_{k2=1}^{N}\delta(-k_1h_1,k_2h_2, 0) E_{m1,m2,m3}(-k_1,k_2,0)\\
\Sigma_{m1,m2,m3}^{+0-}&=\sum_{k1=1}^{N}\sum_{k3=1}^{N}\delta(k_1h_1,0,-k_3h_3) E_{m1,m2,m3}(k_1,0,-k_3\\
\Sigma_{m1,m2,m3}^{-0+}&=\sum_{k1=1}^{N}\sum_{k3=1}^{N}\delta(-k_1h_1,0,k_3h_3) E_{m1,m2,m3}(-k_1,0,k_3\\
\Sigma_{m1,m2,m3}^{0+-}&=\sum_{k2=1}^{N}\sum_{k3=1}^{N}\delta(0, k_2h_2,-k_3h_3) E_{m1,m2,m3}(0, k_2,-k_2)\\
\Sigma_{m1,m2,m3}^{0-+}&=\sum_{k2=1}^{N}\sum_{k3=1}^{N}\delta(0, k_2h_2,-k_3h_3) E_{m1,m2,m3}(0, k_2,-k_3)\\
\Sigma_{m1,m2,m3}^{+00}&=\sum_{k1=1}^{N}\delta(k_1h_1,0,0) E_{m1,m2,m3}(k_1,0,0)\\
\Sigma_{m1,m2,m3}^{-00}&=\sum_{k1=1}^{N}\delta(-k_1h_1,0,0) E_{m1,m2,m3}(-k_1,0,0)\\
\Sigma_{m1,m2,m3}^{0+0}&=\sum_{k2=1}^{N}\delta(0,k_2h_2,0) E_{m1,m2,m3}(0,k_2,0)\\
\Sigma_{m1,m2,m3}^{0-0}&=\sum_{k2=1}^{N}\delta(0,-k_2h_2,0) E_{m1,m2,m3}(0,-k_2,0)\\
\Sigma_{m1,m2,m3}^{00+}&=\sum_{k3=1}^{N}\delta(0,0,k_3h_3) E_{m1,m2,m3}(0,0,k_3)\\
\Sigma_{m1,m2,m3}^{00-}&=\sum_{k3=1}^{N}\delta(0,0,-k_3h_3) E_{m1,m2,m3}(0,0,-k_3)
\end{align*}


\addtocontents{toc}{\protect\setcounter{tocdepth}{4}}
