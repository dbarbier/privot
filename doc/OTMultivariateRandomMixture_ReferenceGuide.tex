% Copyright (c)  2010-2013  EADS.
% Permission is granted to copy, distribute and/or modify this document
% under the terms of the GNU Free Documentation License, Version 1.2
% or any later version published by the Free Software Foundation;
% with no Invariant Sections, no Front-Cover Texts, and no Back-Cover
% Texts.  A copy of the license is included in the section entitled "GNU
% Free Documentation License".




%%%%%%%%%%%%%%%%%%%%%%%%%%%%%%%%%%%%%%%%%%%%%%%%%%%%%%%%%%%%%%%%%%%%%%%%%%%%%%%%%%%%%%%%%%
\section{Reference Guide}

\subsection{Multivariate random mixtures}

The objective is to manipulate a multivariate random vector $\vect{Y}$ with a specific structure:
it is the image of a random vector $\vect{X} = (X_1, \hdots, X_n) $ of independent and continuous univariate distributions by an affine transform $T$:
\begin{align}\label{defY}
  \vect{Y}=\vect{y}_0+\mat{M}\,\vect{X}
\end{align}
$\vect{y}_0\in\mathbb{R}^d$ is a deterministic vector with $d\in\{1,2,3\}$ and $\mat{M}\in\mathcal{M}_{d,n}(\mathbb{R})$
a deterministic matrix. This particular form allows the calculation of the law of $\vect{Y} $ using the Poisson summation formula, which links the characteristic
function $\phi $ of $\vect{Y}$ to its density $p$. This formula is given by:

\begin{align}\label{poissonND}
  \sum_{j_1\in\mathbb{Z}}\hdots\sum_{j_d\in\mathbb{Z}} p\left(y_1+\frac{2\pi j_1}{h_1},\right.&\left.\hdots,y_d+\frac{2\pi j_d}{h_d}\right)=\notag\\
    &\frac{h_1\times\cdots\times h_d}{2^d\pi^d}\sum_{k_1\in\mathbb{Z}}\hdots\sum_{k_d\in\mathbb{Z}}\phi\left(k_1h_1,\hdots,k_dh_d\right)e^{-i(k_1h_1y_1+\cdots+k_dh_dy_d)}
\end{align}

By fixing $h_1,\hdots,h_d$ small enough, the nested sums of the left term are reduced to the central term $j_1=\hdots=j_d = 0$.
The sums on the right are truncated symmetrically to $ 2N +1 $ terms, which gives:
\begin{align}
  p\left(y_1,\hdots,y_d\right)\simeq\frac{H}{2^d\pi^d}\sum_{|k_1|\leq N}\hdots\sum_{|k_d|\leq N}\phi\left(k_1h_1,\hdots,k_dh_d\right)e^{-i(k_1h_1y_1+\cdots+k_dh_dy_d)}
\end{align}
where $H = h_1\times\hdots\times h_d$.

The characteristic function of such a random vector $Y$ is obtained analytically using $\vect{y}_0$, $\mat{M}$ and the characteristic functions of $X_1$,\dots,$X_n$:
\begin{align}
  \forall \vect{u}\in\mathbb{R}^d,\quad\phi(u_1,\hdots,u_d)=\prod_{j=1}^de^{iu_j{y_0}_j}\prod_{k=1}^n\phi_{X_k}((M^tu)_k)
\end{align}

It is possible to greatly improve the performance of the algorithm by noticing that equation~\eqref{poissonND} is linear between $p$
and $\phi $. We denote $q$ and $\psi$ respectively the density and the characteristic function of the multivariate normal distribution with the
same mean $\vect{\mu}$ and same covariance matrix $\vect{C}$ as the random mixture. By applying this multivariate normal distribution to the equation~\eqref{poissonND}, we obtain by subtraction:
\begin{align}\label{algoPoisson}
p\left(y_1,\hdots,y_d\right)\simeq \sum_{j_1\in\mathbb{Z}}\hdots\sum_{j_d\in\mathbb{Z}}
  &q\left(y_1+\frac{2\pi j_1}{h_1},\hdots,y_d+\frac{2\pi j_d}{h_d}\right)+\notag\\
  &\frac{H}{2^d\pi^d}\sum_{|k_1|\leq N}\hdots\sum_{|k_d|\leq N}(\phi-\psi)\left(k_1h_1,\hdots,k_dh_d\right)e^{-i(k_1h_1y_1+\cdots+k_dh_dy_d)}
\end{align}

The vector $\vect{\mu}$ and the covariance matrix $\mat{C}$ of the random vector $\vect{Y}$ are obtained using $\vect{X}$ and the following formula:
\begin{align}
  \vect{\mu}=& \vect{y_0} + \mat{M}\mathbb{E}(\vect{X})\\
  \mat{C}=&\mat{M}\,\mat{\mathrm{Cov}}(\vect{X})\mat{M}^t\\
\end{align}

The matrix $\mat{\mathrm{Cov}}(\vect{X})$ is diagonal thanks to the independence of $X_i$.
The element $C_{i,j}$ of $\mat{C}$ is given by:
\begin{align}
\forall i,j\in\{1,\hdots,d\},\quad C_{i,j}=\sum_{k=1}^nM_{i,k}M_{j,k}\mathrm{Var}(X_k)
\end{align}

In the case where $n \gg $ 1, using the limit central theorem, the law of $\vect{Y} $ tends to the normal distribution density $q$,
which will drastically reduce $N$.  The sum on $q$ will become the most CPU-intensive part, because in the general case we will
have to keep more terms than the central one in this sum, since the parameters $ h_1, \dots  h_d$ were calibrated
with respect to $p$ and not $q$.

The parameters $h_1, \dots  h_d$ are calibrated using the following formula:
\begin{align}
  h_\ell = \frac{2\pi}{(b+4a)\sigma_\ell}
\end{align}
where $\sigma_\ell=\sqrt{C_{\ell,\ell}}$ and $a$, $b$ are respectively the number of standard deviations covered by the marginal distribution
($a=4$ by default) and $b$ the number of marginal deviations beyond which the density is negligible ($b=8$ by default).

The $N$ parameter is dynamically calibrated: we start with $N=8$ then we double $N$ value until the total contribution of the additional terms is negligible.

\subsection{References}\label{ref}
\begin{itemize}
  \item[1] "Abate, J. and Whitt, W. (1992). The Fourier-series method for inverting transforms of probability distributions. Queueing Systems 10, 5--88., 1992",
        formula 5.5.
\end{itemize}

\section{Computation on a regular grid}

The interest is to compute the density function on a regular grid. Purposes are drawing some iso-values.
The regular grid is of form:
\begin{align}
  \forall r\in\{1,\hdots,d\},\forall m\in\{0,\hdots,M-1\},\:y_{r,m}=\mu_r+b\left(\frac{2m+1}{M} - 1\right)\sigma_r
\end{align}

By denoting $p_{m_1,\hdots,m_d}=p_{\vect{Y}}(y_{1,m_1},\hdots,y_{d,m_d})$:
\begin{align}
  p_{m_1,\hdots,m_d}= Q_{m_1,\hdots,m_d}+S_{m_1,\hdots,m_d}
\end{align}
for which the term $S_{m_1,\hdots,m_d}$ is the most CPU consuming. This term rewrites:
\begin{align}
S_{m_1,\hdots,m_d}=&\frac{H}{2^d\pi^d}\sum_{k_1=-N}^{N}\hdots\sum_{k_d=-N}^{N}\delta\left(k_1h_1,\hdots,k_dh_d\right)
E_{m_1,\hdots,m_d}(k_1,\hdots,k_d) \label{Eq:S}
\end{align}
with:
\begin{align}
  \delta\left(k_1h_1,\hdots,k_dh_d\right)&=(\phi-\psi)\left(k_1h_1,\hdots,k_dh_d\right)\\
  E_{m_1,\hdots,m_d}(k_1,\hdots,k_d)&=e^{-i\sum_{j=1}^d k_jh_j\left(\mu_j+b\left(\frac{2m_j+1}{M}-1\right)\sigma_j\right)}
\end{align}

The aim is to rewrite the previous expression as a $d$- discrete Fourier transform, in order to apply Fast Fourier Transform (\emph{FFT}) for its evaluation.

We set $M=N$ and $\forall j \in\{1,\hdots,d\},\: h_j=\frac{\pi}{b\sigma_j}$ and $\tau_j=\frac{\mu_j}{b\sigma_j}$.
For convenience, we introduce the functions:
$$
f_j(k) = e^{-i\pi (k+1)\left(\tau_j-1+\frac{1}{N}\right)}
$$
We use $k+1$ instead of $k$ in this function to simplify expressions below.

We obtain:
\begin{align}
E_{m_1,\hdots,m_d}(k_1,\hdots,k_d)&=e^{-i\sum_{j=1}^{d} k_jh_jb\sigma_j\left(\frac{\mu_j}{b\sigma_j}+\frac{2m_j}{N}+\frac{1}{N}-1\right)}\notag\\
   &=e^{-2i\pi\left(\frac{\sum_{j=1}^{d}k_j m_j}{N}\right)}e^{-i\pi\sum_{j=1}^{d} k_j\left(\tau_j-1+\frac{1}{N}\right)} \notag\\
   &=e^{-2i\pi\left(\frac{\sum_{j=1}^{d}k_j m_j}{N}\right)} f_1(k_1-1) \times\hdots\times f_d(k_d-1) \label{Eq:E}
\end{align}

For performance reasons, we want to use the discrete Fourier transform provided by the numpy Python package.  Its convention is to compute
$$
A_m = \sum_{k=0}^{N-1} a_k e^{-2i\pi\frac{km}{N}}
$$

We decompose sums of~\eqref{Eq:S} on the interval $[-N,N]$ into three parts:
\begin{align}
\sum_{k_j=-N}^{N}\delta\left(k_1h_1,\hdots,k_dh_d\right) E_{m_1,\hdots,m_d}(k_1,\hdots,k_d)
  = & \sum_{k_j=-N}^{-1} \delta\left(k_1h_1,\hdots,k_dh_d\right) E_{m_1,\hdots,m_d}(k_1,\hdots,k_d) \notag\\
  & + \delta\left(k_1h_1,\hdots,0,\hdots,k_dh_d\right) E_{m_1,\hdots,0,\hdots,m_d}(k_1,\hdots,0,\hdots,k_d) \notag\\
  & + \sum_{k_j=1}^{N}\delta\left(k_1h_1,\hdots,k_dh_d\right) E_{m_1,\hdots,m_d}(k_1,\hdots,k_d) \label{Eq:decomposition-sum}
\end{align}

If we already computed $E$ for dimension $d-1$, then the middle term in this sum is trivial.

To compute the last sum of~\eqref{Eq:decomposition-sum}, we apply a change of variable $k_j'=k_j-1$:
\begin{align}
\sum_{k_j=1}^{N}\delta\left(k_1h_1,\hdots,k_dh_d\right) E_{m_1,\hdots,m_d}(k_1,\hdots,k_d)
 = & \sum_{k_j=0}^{N-1}\delta\left(k_1h_1,\hdots,(k_j+1)h_j,\hdots,k_dh_d\right) \times\notag\\
   & \hspace*{3cm} E_{m_1,\hdots,m_d}(k_1,\hdots,k_j+1,\hdots,k_d)
\end{align}
Equation~\eqref{Eq:E} gives:
\begin{align}
E_{m_1,\hdots,m_d}(k_1,\hdots,k_j+1,\hdots,k_d) 
 &= 
     e^{-2i\pi\left(\frac{\sum_{l=1}^{d}k_l m_l}{N} +\frac{m_j}{N}\right)}
     f_1(k_1-1)\times\hdots\times f_j(k_j)\times\hdots\times f_d(k_d-1)\notag\\
 &= 
     e^{-2i\pi\left(\frac{m_j}{N}\right)}
     e^{-2i\pi\left(\frac{\sum_{l=1}^{d}k_l m_l}{N}\right)}
     f_1(k_1-1)\times\hdots\times f_j(k_j)\times\hdots\times f_d(k_d-1)
\end{align}
Thus
\begin{align}
\sum_{k_j=1}^{N}\delta\left(k_1h_1,\hdots,k_dh_d\right) E_{m_1,\hdots,m_d}&(k_1,\hdots,k_d)
  = e^{-2i\pi\left(\frac{m_j}{N}\right)} \sum_{k_j=0}^{N-1}\delta\left(k_1h_1,\hdots,(k_j+1)h_j,\hdots,k_dh_d\right) \times\notag\\
  & e^{-2i\pi\left(\frac{\sum_{l=1}^{d}k_l m_l}{N}\right)}
     f_1(k_1-1)\times\hdots\times f_j(k_j)\times\hdots\times f_d(k_d-1) \label{Eq:j-sigma+}
\end{align}

To compute the first sum of equation~\eqref{Eq:decomposition-sum}, we apply a change of variable $k_j'=N+k_j$:
\begin{align}
\sum_{k_j=-N}^{-1}\delta\left(k_1h_1,\hdots,k_dh_d\right) E_{m_1,\hdots,m_d}(k_1,\hdots,k_d)
 = & \sum_{k_j=0}^{N-1}\delta\left(k_1h_1,\hdots,(k_j-N)h_j,\hdots,k_dh_d\right) \times\notag\\
   & \hspace*{3cm} E_{m_1,\hdots,m_d}(k_1,\hdots,k_j-N,\hdots,k_d)
\end{align}
Equation~\eqref{Eq:E} gives:
\begin{align}
E_{m_1,\hdots,m_d}(k_1,\hdots,k_j-N,\hdots,k_d) 
 &= 
     e^{-2i\pi\left(\frac{\sum_{l=1}^{d}k_l m_l}{N} -m_j\right)}
     f_1(k_1-1)\times\hdots\times f_j(k_j-1-N)\times\hdots\times f_d(k_d-1) \notag\\
 &= 
     e^{-2i\pi\left(\frac{\sum_{l=1}^{d}k_l m_l}{N}\right)}
     f_1(k_1-1)\times\hdots\times \overline{f}_j(N-1-k_j)\times\hdots\times f_d(k_d-1) 
\end{align}

Thus:
\begin{align}
\sum_{k_j=-N}^{-1}\delta\left(k_1h_1,\hdots,k_dh_d\right) E_{m_1,\hdots,m_d}&(k_1,\hdots,k_d)
  = \sum_{k_j=0}^{N-1}\delta\left(k_1h_1,\hdots,(k_j-N)h_j,\hdots,k_dh_d\right) \times\notag\\
  & e^{-2i\pi\left(\frac{\sum_{l=1}^{d}k_l m_l}{N}\right)}
     f_1(k_1-1)\times\hdots\times \overline{f}_j(N-1-k_j)\times\hdots\times f_d(k_d-1) \label{Eq:j-sigma-}
\end{align}

To summarize:
\begin{enumerate}
\item In order to compute sum from $k_1=1$ to $N$, we multiply by $e^{-2i\pi\left(\frac{m_1}{N}\right)}$ and consider $\delta((k_1+1)h,\hdots)f_1(k_1)$
\item In order to compute sum from $k_1=-N$ to $-1$, we consider $\delta((k_1-N)h,\hdots)\overline{f}_1(N-1-k_1)$
\end{enumerate}

\subsection{The unidimensional case}

The interest is to compute
\begin{align}
S_{m}=\frac{h}{2\pi}\Big\{&\Sigma_{m}^{+} + \Sigma_{m}^{-} \Big\}
\end{align}
with:
\begin{align}
\Sigma_{m}^{+}&=\sum_{k=1}^{N}\delta( kh)E_{m}(k) \notag\\
  &=e^{-2i\pi\left(\frac{m}{N}\right)} \sum_{k=0}^{N-1}\left(\delta((k+1)h) f_1(k)\right) e^{-2i\pi\left(\frac{k m}{N}\right)} \\
\Sigma_{m}^{-}&=\sum_{k=-N}^{-1}\delta(kh)E_{m}(k) \notag\\
  &= \sum_{k=0}^{N-1}\left(\delta((k-N)h) \overline{f}_1(N-1-k)\right) e^{-2i\pi\left(\frac{k m}{N}\right)} \notag\\
  &= \sum_{k=0}^{N-1}\left(\overline{\delta}((N-k)h) \overline{f}_1(N-1-k)\right) e^{-2i\pi\left(\frac{k m}{N}\right)}
\end{align}

\begin{enumerate}
\item For $k\in\{0,\hdots,N-1\}$, we set:
\begin{align}
y_{k}=\delta((k+1)h)e^{-i\pi (k+1)\left(\tau-1+\frac{1}{N}\right)}
\end{align}
\item For $m \in\{0,\hdots,N-1\}$, we compute:
\begin{align}
  \widehat{y}_{m}=fft(y_{k})=\sum_{k=0}^{N-1} y_k e^{-2i\pi\left(\frac{k m}{N}\right)} 
\end{align}
with $fft$ an FFT algorithm
\item Eventually $\Sigma_{m}^{+}$ is:
\begin{align*}
  \Sigma_{m}^{+}=\widehat{y}_{m}e^{-2i\pi\left(\frac{m}{N}\right)}
\end{align*}
\item For $k\in\{0,\hdots,N-1\}$,
\begin{align}
z_{k}= \bar{y}_{N-1-k}
\end{align}
\item For $m \in\{0,\hdots,N-1\}$,
\begin{align}
  \Sigma_{m}^{-}=fft(z_{k})
\end{align}
with $fft$ an FFT algorithm
\end{enumerate}

\subsection{The bidimensional case}
In dimension 2, the expression of interest is:
\begin{align}
S_{m_1,m_2}=\frac{h_1h_2}{4\pi^2}\Big\{
  \Sigma_{m_1,m_2}^{++} + \Sigma_{m_1,m_2}^{--} + \Sigma_{m_1,m_2}^{+-} + \Sigma_{m_1,m_2}^{-+} +
  \Sigma_{m_1,m_2}^{+0} + \Sigma_{m_1,m_2}^{-0} + \Sigma_{m_1,m_2}^{0+} + \Sigma_{m_1,m_2}^{0-}
\Big\}
\end{align}
with:
\begin{align}
\Sigma_{m_1,m_2}^{++}&=\sum_{k_1=1}^{N}\sum_{k_2=1}^{N}\delta\left(k_1h_1,k_2h_2\right)E_{m_1,m_2}(k_1,k_2) \label{Eq:sigma++}\\
\Sigma_{m_1,m_2}^{--}&=\sum_{k_1=-N}^{-1}\sum_{k_2=-N}^{-1}\delta\left(k_1h_1,k_2h_2\right)E_{m_1,m_2}(k_1,k_2) \label{Eq:sigma--}\\
\Sigma_{m_1,m_2}^{+-}&=\sum_{k_1=1}^{N}\sum_{k_2=-N}^{-1}\delta\left(k_1h_1,k_2h_2\right)E_{m_1,m_2}(k_1,k_2) \label{Eq:sigma+-}\\
\Sigma_{m_1,m_2}^{-+}&=\sum_{k_1=-N}^{-1}\sum_{k_2=1}^{N}\delta\left(k_1h_1,k_2h_2\right)E_{m_1,m_2}(k_1,k_2) \label{Eq:sigma-+}\\
\Sigma_{m_1,m_2}^{+0}&=\sum_{k_1=1}^{N}\delta(k_1h_1,0)E_{m_1}(k_1) \label{Eq:sigma+0}\\
\Sigma_{m_1,m_2}^{-0}&=\sum_{k_1=-N}^{-1}\delta(k_1h_1,0)E_{m_1}(k_1) \label{Eq:sigma-0}\\
\Sigma_{m_1,m_2}^{0+}&=\sum_{k_2=1}^{N}\delta(0,k_2h_2)E_{m_2}(k_2) \label{Eq:sigma0+}\\
\Sigma_{m_1,m_2}^{0-}&=\sum_{k_2=-N}^{-1}\delta(0,k_2h_2)E_{m_2}(k_2) \label{Eq:sigma0-}
\end{align}

We compute these four arrays for $k_1,k_2\in\{0,\hdots,N-1\}$:
\begin{align*}
y^{++}_{k_1,k_2}&=\delta((k_1+1)h_1,(k_2+1)h_2) f_1(k_1) f_2(k_2) \\
y^{+-}_{k_1,k_2}&=\delta((k_1+1)h_1,(k_2-N)h_2) f_1(k_1) \overline{f}_2(N-1-k_2) \\
y^{+0}_{k_1}&=\delta((k_1+1)h_1,0) f_1(k_1) \\
y^{0+}_{k_2}&=\delta(0,(k_2+1)h_2) f_2(k_2) \\
\end{align*}

\begin{align}
\Sigma_{m_1,m_2}^{++}&=\sum_{k_1=1}^{N}\sum_{k_2=1}^{N}\delta\left(k_1h_1,k_2h_2\right)E_{m_1,m_2}(k_1,k_2) \notag\\
  &=e^{-2i\pi\left(\frac{m_1+m_2}{N}\right)} \sum_{k_1=0}^{N-1}\sum_{k_2=0}^{N-1} \delta\left((k_1+1)h_1,(k_2+1)h_2\right) f_1(k_1) f_2(k_2) e^{-2i\pi\left(\frac{k_1m_1+k_2m_2}{N}\right)} \notag\\
  &=e^{-2i\pi\left(\frac{m_1+m_2}{N}\right)} \sum_{k_1=0}^{N-1}\sum_{k_2=0}^{N-1} y^{++}_{k_1,k_2} e^{-2i\pi\left(\frac{k_1m_1+k_2m_2}{N}\right)} \\
\Sigma_{m_1,m_2}^{--}&=\sum_{k_1=-N}^{-1}\sum_{k_2=-N}^{-1}\delta\left(k_1h_1,k_2h_2\right)E_{m_1,m_2}(k_1,k_2) \notag\\
  &= \sum_{k_1=0}^{N-1}\sum_{k_2=0}^{N-1}\delta\left((k_1-N)h_1,(k_2-N)h_2\right) \overline{f}_1(N-1-k_1)\overline{f}_2(N-1-k_2) e^{-2i\pi\left(\frac{k_1m_1+k_2m_2}{N}\right)} \notag\\
  &= \sum_{k_1=0}^{N-1}\sum_{k_2=0}^{N-1}\overline{\delta\left((N-k_1)h_1,(N-k_2)h_2\right) f_1(N-1-k_1)f_2(N-1-k_2)} e^{-2i\pi\left(\frac{k_1m_1+k_2m_2}{N}\right)} \notag\\
  &= \sum_{k_1=0}^{N-1}\sum_{k_2=0}^{N-1}\overline{y}^{++}_{N-1-k_1,N-1-k_2} e^{-2i\pi\left(\frac{k_1m_1+k_2m_2}{N}\right)} \\
\Sigma_{m_1,m_2}^{+-}&=\sum_{k_1=1}^{N}\sum_{k_2=-N}^{-1}\delta\left(k_1h_1,k_2h_2\right)E_{m_1,m_2}(k_1,k_2) \notag\\
  &=e^{-2i\pi\left(\frac{m_1}{N}\right)} \sum_{k_1=0}^{N-1}\sum_{k_2=0}^{N-1} \delta\left((k_1+1)h_1,(k_2-N)h_2\right) f_1(k_1) \overline{f}_2(N-1-k_2) e^{-2i\pi\left(\frac{k_1m_1+k_2m_2}{N}\right)} \notag\\
  &=e^{-2i\pi\left(\frac{m_1}{N}\right)} \sum_{k_1=0}^{N-1}\sum_{k_2=0}^{N-1} \delta\left((k_1+1)h_1,(k_2-N)h_2\right) f_1(k_1) f_2(k_2-1-N) e^{-2i\pi\left(\frac{k_1m_1+k_2m_2}{N}\right)} \notag\\
  &=e^{-2i\pi\left(\frac{m_1}{N}\right)} \sum_{k_1=0}^{N-1}\sum_{k_2=0}^{N-1} y^{+-}_{k_1,k_2} e^{-2i\pi\left(\frac{k_1m_1+k_2m_2}{N}\right)} \\
\Sigma_{m_1,m_2}^{-+}&=\sum_{k_1=-N}^{-1}\sum_{k_2=1}^{N}\delta\left(k_1h_1,k_2h_2\right)E_{m_1,m_2}(k_1,k_2) \notag\\
  &=e^{-2i\pi\left(\frac{m_2}{N}\right)} \sum_{k_1=0}^{N-1}\sum_{k_2=0}^{N-1} \delta\left((k_1-N)h_1,(k_2+1)h_2\right) \overline{f}_1(N-1-k_1) f_2(k_2) e^{-2i\pi\left(\frac{k_1m_1+k_2m_2}{N}\right)} \notag\\
  &=e^{-2i\pi\left(\frac{m_2}{N}\right)} \sum_{k_1=0}^{N-1}\sum_{k_2=0}^{N-1} \overline{\delta}\left((N-k_1)h_1,-(k_2+1)h_2\right) \overline{f}_1(N-1-k_1) \overline{f}_2(-k_2-2) e^{-2i\pi\left(\frac{k_1m_1+k_2m_2}{N}\right)} \notag\\
  &=e^{-2i\pi\left(\frac{m_2}{N}\right)} \sum_{k_1=0}^{N-1}\sum_{k_2=0}^{N-1} \overline{y}^{+-}_{N-1-k_1,N-1-k_2} e^{-2i\pi\left(\frac{k_1m_1+k_2m_2}{N}\right)} \\
\Sigma_{m_1,m_2}^{+0}&=\sum_{k_1=1}^{N}\delta(k_1h_1,0)E_{m_1}(k_1) \notag\\
  &=e^{-2i\pi\left(\frac{m_1}{N}\right)} \sum_{k_1=0}^{N-1} \delta((k_1+1)h_1,0) f_1(k_1) e^{-2i\pi\left(\frac{k_1m_1}{N}\right)} \notag\\
  &=e^{-2i\pi\left(\frac{m_1}{N}\right)} \sum_{k_1=0}^{N-1} y^{+0}_{k_1} e^{-2i\pi\left(\frac{k_1m_1}{N}\right)}\\
\Sigma_{m_1,m_2}^{-0}&=\sum_{k_1=-N}^{-1}\delta(k_1h_1,0)E_{m_1}(k_1) \notag\\
  &= \sum_{k_1=0}^{N-1} \delta((k_1-N)h_1,0) \overline{f}_1(N-1-k_1) e^{-2i\pi\left(\frac{k_1m_1}{N}\right)}\notag\\
  &= \sum_{k_1=0}^{N-1} \overline{y}^{+0}_{N-1-k_1} e^{-2i\pi\left(\frac{k_1m_1}{N}\right)}\\
\Sigma_{m_1,m_2}^{0+}&=\sum_{k_2=1}^{N}\delta(0,k_2h_2)E_{m_2}(k_2) \notag\\
  &=e^{-2i\pi\left(\frac{m_2}{N}\right)} \sum_{k_2=0}^{N-1} y^{0+}_{k_2} e^{-2i\pi\left(\frac{k_2m_2}{N}\right)}\\
\Sigma_{m_1,m_2}^{0-}&=\sum_{k_2=-N}^{-1}\delta(0,k_2h_2)E_{m_2}(k_2) \notag\\
  &= \sum_{k_2=0}^{N-1} \overline{y}^{0-}_{N-1-k_2} e^{-2i\pi\left(\frac{k_2m_2}{N}\right)}
\end{align}

\subsection{The tridimensional case}
In dimension 3, the expression of interest is~:
\begin{align*}
S_{m_1,m_2,m_3}=\frac{h_1h_2h_3}{8\pi^3}\Big\{
 &   \Sigma_{m_1,m_2,m_3}^{+++} + \Sigma_{m_1,m_2,m_3}^{---} +
     \Sigma_{m_1,m_2,m_3}^{++-} + \Sigma_{m_1,m_2,m_3}^{--+} +
     \Sigma_{m_1,m_2,m_3}^{+-+} + \Sigma_{m_1,m_2,m_3}^{-+-} + \\
 &   \Sigma_{m_1,m_2,m_3}^{+--} + \Sigma_{m_1,m_2,m_3}^{-++} +
     \Sigma_{m_1,m_2,m_3}^{++0} + \Sigma_{m_1,m_2,m_3}^{--0} +
     \Sigma_{m_1,m_2,m_3}^{0++} + \Sigma_{m_1,m_2,m_3}^{0--} + \\
 &   \Sigma_{m_1,m_2,m_3}^{+0+} + \Sigma_{m_1,m_2,m_3}^{-0-} +
     \Sigma_{m_1,m_2,m_3}^{+-0} + \Sigma_{m_1,m_2,m_3}^{-+0} +
     \Sigma_{m_1,m_2,m_3}^{+0-} + \Sigma_{m_1,m_2,m_3}^{-0+} + \\
 &   \Sigma_{m_1,m_2,m_3}^{0+-} + \Sigma_{m_1,m_2,m_3}^{0-+} +
     \Sigma_{m_1,m_2,m_3}^{+00} + \Sigma_{m_1,m_2,m_3}^{-00} +
     \Sigma_{m_1,m_2,m_3}^{0+0} + \Sigma_{m_1,m_2,m_3}^{0-0} + \\
 &   \Sigma_{m_1,m_2,m_3}^{00+} + \Sigma_{m_1,m_2,m_3}^{00-}
\Big\}
\end{align*}

\newcommand{\sigmaThree}[3]{%
  \ifnum#1=-1\relax
     \sum_{k1=-N}^{-1}
  \fi
  \ifnum#1=1\relax
     \sum_{k1=1}^{N}
  \fi
  \ifnum#2=-1\relax
     \sum_{k2=-N}^{-1}
  \fi
  \ifnum#2=1\relax
     \sum_{k2=1}^{N}
  \fi
  \ifnum#3=-1\relax
     \sum_{k3=-N}^{-1}
  \fi
  \ifnum#3=1\relax
     \sum_{k3=1}^{N}
  \fi
  \delta(
    \ifnum#1=0\relax0\else k_1h_1\fi,
    \ifnum#2=0\relax0\else k_2h_2\fi,
    \ifnum#3=0\relax0\else k_3h_3\fi
  ) E_{m_1,m_2,m_3}(
    \ifnum#1=0\relax0\else k_1\fi,
    \ifnum#2=0\relax0\else k_2\fi,
    \ifnum#3=0\relax0\else k_3\fi
  )
}

with
\begin{align*}
\Sigma_{m_1,m_2,m_3}^{+++}&=\sigmaThree{ 1}{ 1}{ 1}\\
\Sigma_{m_1,m_2,m_3}^{---}&=\sigmaThree{-1}{-1}{-1}\\
\Sigma_{m_1,m_2,m_3}^{++-}&=\sigmaThree{ 1}{ 1}{-1}\\
\Sigma_{m_1,m_2,m_3}^{--+}&=\sigmaThree{-1}{-1}{ 1}\\
\Sigma_{m_1,m_2,m_3}^{+-+}&=\sigmaThree{ 1}{-1}{ 1}\\
\Sigma_{m_1,m_2,m_3}^{-+-}&=\sigmaThree{-1}{ 1}{-1}\\
\Sigma_{m_1,m_2,m_3}^{+--}&=\sigmaThree{ 1}{-1}{-1}\\
\Sigma_{m_1,m_2,m_3}^{-++}&=\sigmaThree{-1}{ 1}{ 1}\\
\Sigma_{m_1,m_2,m_3}^{++0}&=\sigmaThree{ 1}{ 1}{ 0}\\
\Sigma_{m_1,m_2,m_3}^{--0}&=\sigmaThree{-1}{-1}{ 0}\\
\Sigma_{m_1,m_2,m_3}^{0++}&=\sigmaThree{ 0}{ 1}{ 1}\\
\Sigma_{m_1,m_2,m_3}^{0--}&=\sigmaThree{ 0}{-1}{-1}\\
\Sigma_{m_1,m_2,m_3}^{+0+}&=\sigmaThree{ 1}{ 0}{ 1}\\
\Sigma_{m_1,m_2,m_3}^{-0-}&=\sigmaThree{-1}{ 0}{-1}\\
\Sigma_{m_1,m_2,m_3}^{+-0}&=\sigmaThree{ 1}{-1}{ 0}\\
\Sigma_{m_1,m_2,m_3}^{-+0}&=\sigmaThree{ 1}{ 1}{ 1}\\
\Sigma_{m_1,m_2,m_3}^{+0-}&=\sigmaThree{ 1}{ 0}{-1}\\
\Sigma_{m_1,m_2,m_3}^{-0+}&=\sigmaThree{-1}{ 0}{ 1}\\
\Sigma_{m_1,m_2,m_3}^{0+-}&=\sigmaThree{ 0}{ 1}{-1}\\
\Sigma_{m_1,m_2,m_3}^{0-+}&=\sigmaThree{ 0}{-1}{ 1}\\
\Sigma_{m_1,m_2,m_3}^{+00}&=\sigmaThree{ 1}{ 0}{ 0}\\
\Sigma_{m_1,m_2,m_3}^{-00}&=\sigmaThree{-1}{ 0}{ 0}\\
\Sigma_{m_1,m_2,m_3}^{0+0}&=\sigmaThree{ 0}{ 1}{ 0}\\
\Sigma_{m_1,m_2,m_3}^{0-0}&=\sigmaThree{ 0}{-1}{ 0}\\
\Sigma_{m_1,m_2,m_3}^{00+}&=\sigmaThree{ 0}{ 0}{ 1}\\
\Sigma_{m_1,m_2,m_3}^{00-}&=\sigmaThree{ 0}{ 0}{-1}
\end{align*}

We compute these thirteen arrays for $k_1,k_2,k_3\in\{0,\hdots,N-1\}$:
\begin{align*}
y^{+++}_{k_1,k_2,k_3}&=\delta((k_1+1)h_1,(k_2+1)h_2,(k_3+1)h_3) f_1(k_1) f_2(k_2) f_3(k_3)\\
y^{++-}_{k_1,k_2,k_3}&=\delta((k_1+1)h_1,(k_2+1)h_2,(k_3-N)h_3) f_1(k_1) f_2(k_2) \overline{f}_3(N-1-k_3)\\
y^{+-+}_{k_1,k_2,k_3}&=\delta((k_1+1)h_1,(k_2-N)h_2,(k_3+1)h_3) f_1(k_1) \overline{f}_2(N-1-k_2) f_3(k_3)\\
y^{+--}_{k_1,k_2,k_3}&=\delta((k_1+1)h_1,(k_2-N)h_2,(k_3-N)h_3) f_1(k_1) \overline{f}_2(N-1-k_2) \overline{f}_3(N-1-k_3)\\
y^{++0}_{k_1,k_2}    &=\delta((k_1+1)h_1,(k_2+1)h_2,         0) f_1(k_1) f_2(k_2)\\
y^{0++}_{k_2,k_3}    &=\delta(         0,(k_2+1)h_2,(k_3+1)h_3) f_2(k_2) f_3(k_3)\\
y^{+0+}_{k_1,k_3}    &=\delta((k_1+1)h_1,0         ,(k_3+1)h_3) f_1(k_1) f_3(k_3)\\
y^{+-0}_{k_1,k_2}    &=\delta((k_1+1)h_1,(k_2-N)h_2,         0) f_1(k_1) \overline{f}_2(N-1-k_2)\\
y^{+0-}_{k_1,k_3}    &=\delta((k_1+1)h_1,0         ,(k_3-N)h_3) f_1(k_1) \overline{f}_3(N-1-k_3)\\
y^{0+-}_{k_2,k_3}    &=\delta(         0,(k_2+1)h_2,(k_3-N)h_3) f_2(k_2) \overline{f}_3(N-1-k_3)\\
y^{+00}_{k_1}        &=\delta((k_1+1)h_1,0         ,         0) f_1(k_1)\\
y^{0+0}_{k_2}        &=\delta(         0,(k_2+1)h_2,         0) f_2(k_2)\\
y^{00+}_{k_3}        &=\delta(         0,0         ,(k_3+1)h_3) f_3(k_3)
\end{align*}


\begin{align*}
\Sigma_{m_1,m_2,m_3}^{+++}&=
  e^{-2i\pi\left(\frac{m_1+m_2+m_3}{N}\right)} \sum_{k_1=0}^{N-1}\sum_{k_2=0}^{N-1}\sum_{k_3=0}^{N-1} y^{+++}_{k_1,k_2,k_3} e^{-2i\pi\left(\frac{k_1m_1+k_2m_2+k_3m_3}{N}\right)} \\
\Sigma_{m_1,m_2,m_3}^{---}&=
  \sum_{k_1=0}^{N-1}\sum_{k_2=0}^{N-1}\sum_{k_3=0}^{N-1} \overline{y}^{+++}_{N-1-k_1,N-1-k_2,N-1-k_3} e^{-2i\pi\left(\frac{k_1m_1+k_2m_2+k_3m_3}{N}\right)} \\
\Sigma_{m_1,m_2,m_3}^{++-}&=
  e^{-2i\pi\left(\frac{m_1+m_2}{N}\right)} \sum_{k_1=0}^{N-1}\sum_{k_2=0}^{N-1}\sum_{k_3=0}^{N-1} y^{++-}_{k_1,k_2,k_3} e^{-2i\pi\left(\frac{k_1m_1+k_2m_2+k_3m_3}{N}\right)} \\
\Sigma_{m_1,m_2,m_3}^{--+}&=
  e^{-2i\pi\left(\frac{m_3}{N}\right)} \sum_{k_1=0}^{N-1}\sum_{k_2=0}^{N-1}\sum_{k_3=0}^{N-1} \overline{y}^{++-}_{N-1-k_1,N-1-k_2,N-1-k_3} e^{-2i\pi\left(\frac{k_1m_1+k_2m_2+k_3m_3}{N}\right)} \\
\Sigma_{m_1,m_2,m_3}^{+-+}&=
  e^{-2i\pi\left(\frac{m_1+m_3}{N}\right)} \sum_{k_1=0}^{N-1}\sum_{k_2=0}^{N-1}\sum_{k_3=0}^{N-1} y^{+-+}_{k_1,k_2,k_3} e^{-2i\pi\left(\frac{k_1m_1+k_2m_2+k_3m_3}{N}\right)} \\
\Sigma_{m_1,m_2,m_3}^{-+-}&=
  e^{-2i\pi\left(\frac{m_2}{N}\right)} \sum_{k_1=0}^{N-1}\sum_{k_2=0}^{N-1}\sum_{k_3=0}^{N-1} \overline{y}^{+-+}_{N-1-k_1,N-1-k_2,N-1-k_3} e^{-2i\pi\left(\frac{k_1m_1+k_2m_2+k_3m_3}{N}\right)} \\
\Sigma_{m_1,m_2,m_3}^{+--}&=
  e^{-2i\pi\left(\frac{m_1}{N}\right)} \sum_{k_1=0}^{N-1}\sum_{k_2=0}^{N-1}\sum_{k_3=0}^{N-1} y^{+--}_{k_1,k_2,k_3} e^{-2i\pi\left(\frac{k_1m_1+k_2m_2+k_3m_3}{N}\right)} \\
\Sigma_{m_1,m_2,m_3}^{-++}&=
  e^{-2i\pi\left(\frac{m_2+m_3}{N}\right)} \sum_{k_1=0}^{N-1}\sum_{k_2=0}^{N-1}\sum_{k_3=0}^{N-1} \overline{y}^{+--}_{N-1-k_1,N-1-k_2,N-1-k_3} e^{-2i\pi\left(\frac{k_1m_1+k_2m_2+k_3m_3}{N}\right)} \\
\Sigma_{m_1,m_2,m_3}^{++0}&=
  e^{-2i\pi\left(\frac{m_1+m_2}{N}\right)} \sum_{k_1=0}^{N-1}\sum_{k_2=0}^{N-1} y^{++0}_{k_1,k_2} e^{-2i\pi\left(\frac{k_1m_1+k_2m_2}{N}\right)} \\
\Sigma_{m_1,m_2,m_3}^{--0}&=
  \sum_{k_1=0}^{N-1}\sum_{k_2=0}^{N-1} \overline{y}^{++0}_{N-1-k_1,N-1-k_2} e^{-2i\pi\left(\frac{k_1m_1+k_2m_2}{N}\right)} \\
\Sigma_{m_1,m_2,m_3}^{0++}&=
  e^{-2i\pi\left(\frac{m_2+m_3}{N}\right)} \sum_{k_2=0}^{N-1}\sum_{k_3=0}^{N-1} y^{0++}_{k_2,k_3} e^{-2i\pi\left(\frac{k_2m_2+k_3m_3}{N}\right)} \\
\Sigma_{m_1,m_2,m_3}^{0--}&=
  \sum_{k_2=0}^{N-1}\sum_{k_3=0}^{N-1} \overline{y}^{0++}_{N-1-k_2,N-1-k_3} e^{-2i\pi\left(\frac{k_2m_2+k_3m_3}{N}\right)} \\
\Sigma_{m_1,m_2,m_3}^{+0+}&=
  e^{-2i\pi\left(\frac{m_1+m_3}{N}\right)} \sum_{k_1=0}^{N-1}\sum_{k_3=0}^{N-1} y^{+0+}_{k_1,k_3} e^{-2i\pi\left(\frac{k_1m_1+k_3m_3}{N}\right)} \\
\Sigma_{m_1,m_2,m_3}^{-0-}&=
  \sum_{k_1=0}^{N-1}\sum_{k_3=0}^{N-1} \overline{y}^{0++}_{N-1-k_1,N-1-k_3} e^{-2i\pi\left(\frac{k_1m_1+k_3m_3}{N}\right)} \\
\Sigma_{m_1,m_2,m_3}^{+-0}&=
  e^{-2i\pi\left(\frac{m_1}{N}\right)} \sum_{k_1=0}^{N-1}\sum_{k_2=0}^{N-1} y^{+-0}_{k_1,k_2} e^{-2i\pi\left(\frac{k_1m_1+k_2m_2}{N}\right)} \\
\Sigma_{m_1,m_2,m_3}^{-+0}&=
  e^{-2i\pi\left(\frac{m_2}{N}\right)} \sum_{k_1=0}^{N-1}\sum_{k_2=0}^{N-1} \overline{y}^{+-0}_{N-1-k_1,N-1-k_2} e^{-2i\pi\left(\frac{k_1m_1+k_2m_2}{N}\right)} \\
\Sigma_{m_1,m_2,m_3}^{+0-}&=
  e^{-2i\pi\left(\frac{m_1}{N}\right)} \sum_{k_1=0}^{N-1}\sum_{k_3=0}^{N-1} y^{+0-}_{k_1,k_3} e^{-2i\pi\left(\frac{k_1m_1+k_3m_3}{N}\right)} \\
\Sigma_{m_1,m_2,m_3}^{-0+}&=
  e^{-2i\pi\left(\frac{m_3}{N}\right)} \sum_{k_1=0}^{N-1}\sum_{k_3=0}^{N-1} \overline{y}^{+0-}_{N-1-k_1,N-1-k_3} e^{-2i\pi\left(\frac{k_1m_1+k_3m_3}{N}\right)} \\
\Sigma_{m_1,m_2,m_3}^{0+-}&=
  e^{-2i\pi\left(\frac{m_2}{N}\right)} \sum_{k_2=0}^{N-1}\sum_{k_3=0}^{N-1} y^{0+-}_{k_2,k_3} e^{-2i\pi\left(\frac{k_2m_2+k_3m_3}{N}\right)} \\
\Sigma_{m_1,m_2,m_3}^{0-+}&=
  e^{-2i\pi\left(\frac{m_2}{N}\right)} \sum_{k_2=0}^{N-1}\sum_{k_3=0}^{N-1} \overline{y}^{0+-}_{N-1-k_2,N-1-k_3} e^{-2i\pi\left(\frac{k_2m_2+k_3m_3}{N}\right)} \\
\Sigma_{m_1,m_2,m_3}^{+00}&=
  e^{-2i\pi\left(\frac{m_1}{N}\right)} \sum_{k_1=0}^{N-1} y^{+00}_{k_1} e^{-2i\pi\left(\frac{k_1m_1}{N}\right)} \\
\Sigma_{m_1,m_2,m_3}^{-00}&=
  \sum_{k_1=0}^{N-1} \overline{y}^{+00}_{k_1} e^{-2i\pi\left(\frac{k_1m_1}{N}\right)} \\
\Sigma_{m_1,m_2,m_3}^{0+0}&=
  e^{-2i\pi\left(\frac{m_2}{N}\right)} \sum_{k_2=0}^{N-1} y^{0+0}_{k_2} e^{-2i\pi\left(\frac{k_2m_2}{N}\right)} \\
\Sigma_{m_1,m_2,m_3}^{0-0}&=
  \sum_{k_2=0}^{N-1} \overline{y}^{0+0}_{k_2} e^{-2i\pi\left(\frac{k_2m_2}{N}\right)} \\
\Sigma_{m_1,m_2,m_3}^{00+}&=
  e^{-2i\pi\left(\frac{m_3}{N}\right)} \sum_{k_3=0}^{N-1} y^{00+}_{k_3} e^{-2i\pi\left(\frac{k_3m_3}{N}\right)} \\
\Sigma_{m_1,m_2,m_3}^{00-}&=
  \sum_{k_3=0}^{N-1} \overline{y}^{00+}_{k_3} e^{-2i\pi\left(\frac{k_3m_3}{N}\right)} \\
\end{align*}

